\documentclass[11pt, oneside]{amsart}
\pdfoutput=1

\usepackage{amsmath}
\usepackage{amssymb}

\usepackage[table]{xcolor}
\usepackage{dcolumn}
\usepackage{float}
\usepackage{graphicx}
\usepackage[utf8]{inputenc}
\usepackage[T1]{fontenc}
\usepackage{lmodern}
\usepackage{multirow}
\usepackage{rotating}
\usepackage{subfigure}
\usepackage{psfrag}
\usepackage{tabularx}
\usepackage[hyphens]{url}
\usepackage{wrapfig}
\usepackage{longtable}
\usepackage{verbatim}
\usepackage{booktabs,multicol}

% The following three lines are used for displaying footnote in tables.
\usepackage{footnote}
\makesavenoteenv{tabular}
\makesavenoteenv{table}

\usepackage{enumitem}
\setlist{leftmargin=7mm}

\usepackage[bookmarks, bookmarksopen, bookmarksnumbered]{hyperref}
\usepackage[all]{hypcap}
\urlstyle{rm}

\definecolor{darkgreen}{rgb}{0,0.6,0}
\newcommand{\katznote}[1]{ {\textcolor{blue} { ***DSK: #1 }}}
\newcommand{\niemnote}[1]{ {\textcolor{orange} { ***KEN: #1 }}}
\newcommand{\asnote}[1]{ {\textcolor{red} { ***AMS: #1 }}}
\newcommand{\flnote}[1]{ {\textcolor{olive} { ***FL: #1 }}}
\newcommand{\scnote}[1]{ {\textcolor{darkgreen} { ***SCC: #1 }}} % Sou-Cheng Choi
\newcommand{\dmnote}[1]{ {\textcolor{purple} { ***DM: #1 }}} % Daniel Mietchen
\newcommand{\acmnote}[1]{ {\textcolor{green} { ***ACM: #1 }}} % Abigail Cabunoc Mayes 

% 15 characters / 2.5 cm => 100 characters / line
% Using 11 pt => 94 characters / line
\setlength{\paperwidth}{216 mm}
% 6 lines / 2.5 cm => 55 lines / page
% Using 11pt => 48 lines / pages
\setlength{\paperheight}{279 mm}
\usepackage[top=2.5cm, bottom=2.5cm, left=2.5cm, right=2.5cm]{geometry}
% You can use a baselinestretch of down to 0.9
\renewcommand{\baselinestretch}{0.96}



\title{Software Citation Principles}

\author{Force11 Software Citation Working Group (Editors: Arfon M. Smith, Daniel S. Katz, Kyle E. Niemeyer)}

\date{}

\begin{document}

\begin{abstract}
\end{abstract}

\maketitle


%%%%%%%%%%%%%%%%%%%%%%%%%%%%%%%%%%%%%%%%%%%%%%%%%%%%%%%%%%%%
\section{Introduction \& motivation}
\label{sec:intro}
%%%%%%%%%%%%%%%%%%%%%%%%%%%%%%%%%%%%%%%%%%%%%%%%%%%%%%%%%%%%

As the process of science has become increasingly digital, scientific outputs and products have
grown beyond simple papers and books to include software, data, and other electronic
components.  Scientific knowledge is embedded in these components.  And papers and books
themselves are also becoming increasingly digital, allowing them to become executable
and reproducible.  As we move towards this future where science is performed in and recorded
as a variety of linked digital products, the characteristics and properties that developed for
books and papers need to be applied to all digital products and possibly adjusted.  Here, we are concerned specifically
with the citation of software products.

\katznote{add cites to James's study how what people cite software and Matthew Mayernik's similar one as part of motivation} 

The Force11 Software Citation Working Group~\cite{f11scwg} was created in April 2015, with the following mission statement:

\begin{quote}
The software citation working group is a committee that will leverage the perspectives of a variety of existing initiatives working on software citation to produce a consolidated set of citation principles in order to encourage broad adoption of a consistent policy for software citation across disciplines and venues. The working group will review existing efforts and make a set of recommendations. These recommendations will be put up for endorsement by the organizations represented by this group and others that play an important role in the community.

The group will produce a set of principles, illustrated with working examples, and a plan for dissemination and distribution. This group will not be producing detailed specifications for implementation although it may review and discuss possible technical solutions.\end{quote}

The group gathered members (see Appendix~\ref{app:wg_members}) in April and May 2015, and then in June began work, with a number of meetings
and some off-line work by group members to gather materials documenting existing
practices in member disciplines; gather materials from workshops and other reports;
review those materials, identifying overlaps and differences; and draft this resulting document,
which will be presented and discussed at the Force2016 Conference \cite{force2016}
in April 2016.  We expect that
this discussion may lead to a second, final version, and we also
plan to have a follow-on working group that will work with stakeholders to ensure that
these principles impact the scientific process.

The principles in this document should guide further development of software citation mechanisms and system, and the reader should be able to look at any particular example of software citation and see if it meets the principles.

In the next section (\S\ref{sec:use_cases}), we provide some detailed context in which
software citation is important, by means of use cases.  In \S\ref{sec:related_work}, we
summarize and analyze a large amount of previous work and thinking in this area.  In
\S\ref{sec:principles}, we provide a set of guiding principles for citation of software within
scholarly literature, other software, datasets, or any other research object.  Finally,
in \S\ref{sec:examples} we give examples of how these software citation principles
could be applied, related back to the use cases in \S\ref{sec:use_cases}.

 \acmnote{would like to add info on the problem we're trying to solve in academia to motivation} 
 
%%%%%%%%%%%%%%%%%%%%%%%%%%%%%%%%%%%%%%%%%%%%%%%%%%%%%%%%%%%%
\section{Use cases}
\label{sec:use_cases}
%%%%%%%%%%%%%%%%%%%%%%%%%%%%%%%%%%%%%%%%%%%%%%%%%%%%%%%%%%%%

We have documented and analyzed a set of use cases related to software citation in \cite{SC-Use-Cases}.
Table~\ref{tab:use_cases} summarizes these use cases and makes clear what the requirements are for software citation in each case.
\katznote{not sure if this next long sentence should be here, should be in the table caption, or should be footnotes to the table, or even more than one of these}
\niemnote{Not sure either; definitely should include in the text due to length. Maybe an abbreviated version in the caption?}\flnote{If these ``definitions'' are not used anywhere else than in the table (and they don't really seem to be so far), then I tend towards putting this in the caption. That's where they are needed (in the table). I don't like footnotes for something like this.}\dmnote{Agree it would be better in the caption.}
In the table,
``researcher'' includes both academic researchers (e.g., postdoc, tenure-track faculty member) and research software engineers;
``publisher'' includes both traditional publishers that publish text and\slash or software papers as well as archives such as Zenodo that directly publish software;
``reproduce'' can mean reproduction, replication, verification, validation, repeatability, and\slash or utility;
examples of indexers include Scopus, Web of Science, Google Scholar, and Microsoft Academic Search;
domain groups include bioCADDIE, CIG, etc.;\dmnote{Not sure what CIG stands for, and acronyms should probably be explained or linked}
``repository'' refers to open-source software repositories such as ASCL, GitHub, Netlib, CRAN, institutional repositories;
``funder'' is a group that funds software or work using software; and
``UID'' means unique ID, such as a DOI or PURL.

\rowcolors{5}{white}{gray!25}
\begin{table}[htbp]
\caption{Use cases for software citation, adapted from \cite{SC-Use-Cases} \flnote{Also, I noticed that some entries have the same requirements: maybe they could be combined to one line?} \dmnote{I would leave them separate. What about adding a column for license?}}
\centering
\scriptsize\setlength{\tabcolsep}{2.5pt}
\begin{tabular}{@{}l l c c c c c c c c@{}}
\toprule
 & & \multicolumn{6}{c}{Requirements} \\
 \cmidrule{3-10}
\multirow{2}{*}{Stakeholder} &	\multirow{2}{*}{Use\slash wants to} 	 &  Software  & \multirow{2}{*}{Author(s)} & \multirow{2}{*}{Version \#} & Release & Location\slash  & \multirow{2}{*}{UID} & Indexed & \multirow{2}{*}{Role} \\
& & name &  &  &  date & repository &  & citations & \\
\midrule
Researcher            & use someone else's software for a paper      & \textbullet & \textbullet & \textbullet & \textbullet & \textbullet & \textbullet &             &             \\
Researcher            & use someone else's software for new software & \textbullet & \textbullet & \textbullet & \textbullet & \textbullet & \textbullet &             &             \\
Researcher            & contribute to software                   & \textbullet & \textbullet & \textbullet & \textbullet & \textbullet & \textbullet &             & \textbullet \\
Researcher            & find citations of software               & \textbullet &             &             &             &             & \textbullet & \textbullet &             \\
Researcher            & get credit for software development          & \textbullet & \textbullet &             & \textbullet & \textbullet & \textbullet &             & \textbullet \\
Researcher            & ``reproduce'' analysis                   & \textbullet &             & \textbullet & \textbullet & \textbullet & \textbullet &             &             \\
Researcher            & benchmark software                       & \textbullet &             & \textbullet & \textbullet & \textbullet & \textbullet &             &             \\
Researcher            & find software to implement task          & \textbullet & \textbullet &             &             & \textbullet & \textbullet & \textbullet &             \\
Publisher             & publish software paper                   & \textbullet & \textbullet & \textbullet & \textbullet & \textbullet & \textbullet &             &             \\
Publisher             & publish papers that cite software        & \textbullet & \textbullet & \textbullet & \textbullet & \textbullet & \textbullet & \textbullet &             \\
Indexer               & build catalog of software                & \textbullet & \textbullet & \textbullet & \textbullet & \textbullet & \textbullet & \textbullet &             \\
Domain group          & build catalog of software                & \textbullet & \textbullet & \textbullet & \textbullet & \textbullet & \textbullet &             &             \\
Library\slash archive & build catalog of software                & \textbullet & \textbullet & \textbullet & \textbullet & \textbullet & \textbullet &             &             \\
Repository            & show scientific impact of holdings       & \textbullet &             &             &             &             & \textbullet & \textbullet &             \\
Funder                & show how funded software has been used   & \textbullet &             &             &             &             & \textbullet & \textbullet &             \\
\bottomrule
\end{tabular}
\label{tab:use_cases}
\end{table}%

If certain metadata are not available, alternatives may be provided:
\begin{itemize}
\item version number and release date: download date
\item location\slash repository: contact name\slash email if not publicly
  available
\end{itemize}




%%%%%%%%%%%%%%%%%%%%%%%%%%%%%%%%%%%%%%%%%%%%%%%%%%%%%%%%%%%%
\section{Related work}
\label{sec:related_work}
%%%%%%%%%%%%%%%%%%%%%%%%%%%%%%%%%%%%%%%%%%%%%%%%%%%%%%%%%%%%

With close to 50 working group participants representing a range of research domains, the working group was tasked to
document existing practices in their respective communities. A total of 47 documents were submitted by working group
participants, with the life sciences, astrophysics, and geosciences being particularly well-represented in the submitted
resources.

\textbf{Summary of submitted resources}

\asnote{Should mention that a number of communities have already focussed their attention on issues around data citation and while
this isn't exactly the same there are many similarities (what is different exactly?).}

\textbf{General community/non domain-specific activities}

Some of the most actionable work has come from the NSF Software Infrastructure for Sustained Innovation (SSI) program in the form of blog posts written by their community fellows:

In a blog post from 2012, \cite{ssi-how-to-cite} discusses some of the pitfalls of trying to cite software in publications.
The author includes useful guidance for when to consider citing software as well as some ways to help `convince' journal editors
to allow the inclusion of software citations.

In \cite{ssi-citation-files}, Wilson suggests that software authors include a `CITATION' file that documents exactly how the
authors of the software would like to be cited by others. While this isn't a formal metadata specification (it's not machine
readable for example) this does offer a solution for authors wishing to give explicit instructions to potential citing authors.\dmnote{There is evidence from databases that people actually tend to follow such instructions if they exist: \href{http://doi.org/10.1371/journal.pone.0136631}{10.1371/journal.pone.0136631}.}
In a later post on the SSI blog, Jackson \cite{ssi-how-shalt-i-cite-thee} gives a good overview of some of the approaches
package authors have taken to automate the generation of citation entities such as bibtex entries.
\scnote{So does \cite{knepley2013accurately}}

Perhaps in recognition of the broad range of research domains struggling with
the challenge of better recognizing the role of software, a number of community
efforts hosted (and sponsored) by funders and agencies in both the US (e.g., NSF,
NIH, Alfred P. Sloan Foundation) and UK (e.g., SFTC, JISC) have run a number of
workshops with participants from across a range of disciplines.

\asnote{Should also discuss the WSSSPE \cite{wssspe} series of workshops in this section}
\scnote{Section 5.8 and Appendix J in \cite{WSSSPE3} has some preliminary work and discussion on the topic.}

\textbf{Domain-specific community activities}

One approach to increasing the `citability' of software is to encourage the
submission of papers in standard journals describing a piece of research
software. While some journals (e.g., F1000Research) have traditionally accepted
software submissions, the American Astronomical Society (AAS) have recently
announced they will accept software papers in their journals
\cite{aas-sofware-papers}.

In astronomy \& astrophysics: ASCL \cite{ascl}, AAS journals  AAS/GitHub
workshop \cite{aas-software-index} dedicated to software citation, indexing and discoverability.

\asnote{Perhaps we should say something here about the potential importance of professional societies and their role in effecting changes in community practice?}

Life sciences/NIH: Software Discovery Index Meeting Report \cite{software-discovery-index} a cross-disciplinary meeting
but with a largely NIH-focus. Suggested minimal information about software (MIAS) standard. Discusses how a common set of metadata
fields are critical for useful indexing - suggested candidates for this list.

Geosciences: Ontosoft \cite{ontosoft} - A Community Software Commons for the Geosciences. Lots of attention given to the metadata
required to describe, discover and execute research software. NSF geosciences workshop around data lifecycle, management and citation \cite{nsf-geo-data}. This report includes many recommendations for data citation.

\textbf{Existing principles}

Author guidelines in F1000Research, APA authors, SSI `publish or be damned' \cite{ssi-publish-or-be-damned} manifesto. \dmnote{There are many other journals publishing software descriptions, some even specialized in software, e.g. \href{http://www.openresearchcomputation.com}{Open Research Computation}.}

\textbf{Existing efforts around metadata standards}

AAS workshop (\cite{aas-software-index}), CRediT project \cite{casrai-credit}, Ontosoft \cite{ontosoft}

\textbf{Studies of author motivations}

Survey of data citation practices \cite{Kratz_2015}

%%%%%%%%%%%%%%%%%%%%%%%%%%%%%%%%%%%%%%%%%%%%%%%%%%%%%%%%%%%%
\section{Software citation principles}
\label{sec:principles}
%%%%%%%%%%%%%%%%%%%%%%%%%%%%%%%%%%%%%%%%%%%%%%%%%%%%%%%%%%%%

\katznote{The current content of this section is liberally borrowed and loosely adapted from \cite{data-citation-principles}.  We we will want to add some text here explaining that we started with that document, and then modified it to take into account the information from the use cases and the related work, once we do so.}

\begin{enumerate}
\item \textbf{Importance}:
Software should be considered a legitimate, citable product of research. Software citations should be accorded the same importance in the scholarly record as citations of other research objects, such as publications and data. Software should be cited whenever and wherever a research product (such as a paper) relies upon it, specifically, as part of the standard reference list for that research product.
\item \textbf{Credit and Attribution}
Software citations should facilitate giving scholarly credit and normative and legal attribution to all contributors to the software, recognizing that a single style or mechanism of attribution may not be applicable to all software.\scnote{At least four styles (AMS, APA, Chicago, MLA) for software citation are illustrated with examples in ~\cite{lipson2011cite}.}
\item \textbf{Unique Identification}
A software citation should include a persistent method for identification that is machine actionable, globally unique, and widely used by a community.
\item \textbf{Persistence}
Unique identifiers, and metadata describing the software, and its disposition, should persist -- even beyond the lifespan of the software they describe.
\item \textbf{Accessibility}
Software citations should facilitate access to the software itself and to such associated metadata, documentation, data, and other materials, as are necessary for both humans and machines to make informed use of the referenced software.\dmnote{Should provide a location where the software is available. Should we say anything about environments?}
\item \textbf{Specificity and Verifiability}
Software citations should facilitate identification of, access to, and verification of the specific software that support a claim. Citations or citation metadata should include information about provenance and fixity sufficient to facilitate verifying that the specific version and/or portion of software retrieved subsequently is the same as was originally cited.
\item \textbf{Interoperability and Flexibility}
Software citation methods should be sufficiently flexible to accommodate the variant practices among communities, but should not differ so much that they compromise interoperability of software citation practices across communities.
\item \textbf{Anything missing?}
Interactive software? Screenshots? \ldots ?
\end{enumerate}


%%%%%%%%%%%%%%%%%%%%%%%%%%%%%%%%%%%%%%%%%%%%%%%%%%%%%%%%%%%%
\section{Implementation examples}
\label{sec:examples}
%%%%%%%%%%%%%%%%%%%%%%%%%%%%%%%%%%%%%%%%%%%%%%%%%%%%%%%%%%%%

\appendix

\section{Working Group Membership}
\label{app:wg_members}

\katznote{should make affiliation styles more consistent}

\katznote{should check to make sure these are all still right, such as for me and Bill Mills}

Alberto Accomazzi, Harvard-Smithsonian CfA

Alice Allen, Astrophysics Source Code Library

Carl Boettiger, UC Berkeley

Sou-Cheng Choi, NORC at the University of Chicago \& Illinois Institute of Technology	

Neil Chue Hong, Software Sustainability Institute	

Tom Crick, Cardiff Metropolitan University	

Martin Fenner, Public Library of Science	

Merc\`e Crosas, IQSS, Harvard University	

Christopher Erdmann, Harvard-Smithsonian CfA	

Ian Gent, University of St Andrews, recomputation.org	

Paul Groth, Elsevier Labs	

Melissa Haendel, OHSU	

Stephanie Hagstrom, Force11	

Robert Hanisch, NIST/ODI	

Edwin Henneken, Harvard-Smithsonian CfA	

Ivan Herman, W3C	

James Howison, UTexas	

Lorraine Hwang, UC Davis	

Thomas Ingraham, F1000Research	

Matthew B. Jones, NCEAS, UC Santa Barbara	

Catherine Jones, Science and Technology Facilities Council

Daniel S. Katz, U Chicago \& Argonne Natl Lab (co-chair)

Alexander Konovalov, University of St Andrews

John Kratz, California Digital Library	

Jennifer Lin, Public Library of Science	

Frank L\"offler, Louisiana State University	

Brian Matthews, Science and Technology Facilites Council	

Abigail Cabunoc Mayes, Mozilla Science Lab	

Daniel Mietchen, National Institutes of Health	

Bill Mills, Mozilla Science Lab	

Evan Misshula, CUNY Graduate Center	

August Muench, American Astronomical Society	

Fiona Murphy, Independent Researcher	

Lars Holm Nielsen, CERN	

Kyle Niemeyer, Oregon State University	

Karthik Ram, University of California, Berkeley	

Fernando Rios, Johns Hopkins University	

Ashley Sands, UCLA Information Studies	

Soren Scott , Independent Researcher

Frank J. Seinstra, Netherlands eScience Center

Arfon Smith, GitHub (co-chair)

Kaitlin Thaney, Mozilla Science Lab

Ilian Todorov, STFC

Matt Turk, University of Illinois

Miguel de Val-Borro, Princeton University

Daan Van Hauwermeiren, Ghent University

Stijn Van Hoey, Ghent University

Belinda Weaver, The University of Queensland

Nic Weber, University of Washington iSchool

\bibliographystyle{abbrv}
\bibliography{software-citation-principles}


\end{document}
